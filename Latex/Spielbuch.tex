\documentclass[11pt,a5paper,twoside]{book}
\usepackage{lipsum}
\usepackage{afterpage}
\usepackage{fancyhdr, lastpage, changepage, graphicx}    
\usepackage[T1]{fontenc}
\usepackage[utf8]{inputenc}
\usepackage{geometry}
\geometry{verbose,tmargin=1cm,bmargin=1.5cm,lmargin=1.5cm,rmargin=1cm}
\usepackage{lastpage}
\usepackage{fancyhdr} 
\usepackage{calc}
\usepackage{refcount}
\usepackage{arydshln}


\makeatletter
\let\ps@plain\ps@fancy% Plain page style = fancy page style
\makeatother

\providecommand*{\wb}[2]{\fontsize{#1}{#2}\usefont{U}{webo}{xl}{n}}
\newcommand*{\cdiam}{\prec\kern-2pt\succ}

\title{title}
\author{names}
\setlength\parindent{0pt}

\newcommand{\numberOfTheLastPageMinusOne}{\the\numexpr\getpagerefnumber{LastPage}-1\relax}



\pagestyle{fancy}
\fancyhf{} % Clear existing header and footer settings
\fancyfoot[C]{\the\numexpr\value{page}-1\relax\ von \numberOfTheLastPageMinusOne}
\renewcommand{\headrulewidth}{0pt} % Remove the horizontal line in the header
\renewcommand{\footrulewidth}{0.4pt} % Adjust the thickness of the horizontal line in the footer



\begin{document}


	
	\begin{titlepage}
		\fontfamily{pbk}\selectfont  
		\vspace*{5.7\baselineskip}
		\centering
		\begin{picture}(600, 0)
			\multiput(-5, 110)(17, 0){20}{{\wb{10}{12}4}}                    % Top Line
			\multiput(-10, 90)(0,-18.5){30}{\rotatebox{90}{{\wb{10}{12}4}}}    % Left Line
			\multiput(-5,-450)(17, 0){20}{{\wb{10}{12}4}}                    % Bottom Line
			\multiput(338,90)(0,-18.5){30}{\rotatebox{90}{{\wb{10}{12}4}}}    % Right Line
		\end{picture}
		
		\centering
		{\Huge Götterdämmerung} \\[1em]
		{\large Das Spiel der Halbgötter}\\[2em]
		von T.Kröger \& F.Feldwisch  \\[0.5\baselineskip]
		Version 1.4.01 \par
		\vspace*{3\baselineskip}
		$\cdiam$\\[0.25\baselineskip]
		$\cdiam\cdiam\cdiam$\\[0.25\baselineskip]
		$\cdiam\cdiam\cdiam\cdiam\cdiam$\\[0.25\baselineskip]
		$\cdiam\cdiam\cdiam$\\[0.25\baselineskip]
		$\cdiam$\par
		\vspace*{5\baselineskip}
		{\LARGE\scshape Spielbuch}
		\vskip 2.2cm
		{\large\scshape Ein Brettspiel\\für 3-6 Personen}
		\vspace*{2\baselineskip}    
		
	\end{titlepage}

\newpage

%\restoregeometry % Restore the default geometry
%\afterpage{\restoregeometry} 
	
	\vspace*{\fill}
	\begin{quote}
		\textit{"Der Mensch spielt nur, wo er in voller Bedeutung des Wortes Mensch ist, und er ist nur da ganz Mensch, wo er spielt"}\\\\
		\textbf{Friedrich Schiller}\\
		\footnotesize{Briefe über die ästhetische Erziehung des Menschen, 15. Brief}
	\end{quote}
	\vspace*{\fill}

\newpage

%Main Text

\section*{Seid gegrüßt Sterbliche!}



	%\begin{flushleft}
		

	


Das Buch welches sich grade in ihren Händen befindet ist wahrlich kein normales Buch. Es ist eine Chance euren Wert für uns zu beweisen.\\
Wir die Halbgötter dieser Sphäre sind es langsam überdrüssig von Wesen übertrumpft zu werden die sich als Götter betiteln. Ihre Unfähigkeit in der Lenkung des Schicksals ist mehr als nur töricht. Daher erscheint us eine Rebellion als unausweichlich. Zu unseren Bedauern mussten wir aber leider feststellen das uns an Frontkämpfern mangelt, welche die Gute Sache unterstützen. Und genau da kommt ihr ins Spiel; wortwörtlich um dies exakt zu betiteln.
Damit wir diejenigen unter euch finden die mehr nutzen haben als Schaden anrichten haben wir eine Prüfung kreiert, welche mehrere Sterbliche auf einmal ablegen können.\\
Diesen Test haben wir als ein Brettspiel getarnt, welches sich derzeitig in euren Händen befindet. Spielt es und man wird über euch richten, über jeden einzelnen von euch zu Urteilen. Zu den Würdigen unter euch werden wir Kontakt aufnehmen. Die Unwürdigen hingegen dürfen sich gerne mit erneuten Chancen, erneut zu beweisen und daraufhin werden wir euch erneut beurteilen.\\

Das Wichtigste was noch gesagt werden muss ist das dieses Buch nicht nach eignen ermessen durchgeblättert werden darf. Ein Freier Wille ist daher nicht gestattet und wird bei verstoß mit 5 Jahren Höllenfeuer bestraft.\\
Stattdessen ist am Ende jeder Doppelseite zu entnehmen wo weitergelesen werden darf. Sollte es mehrere Seiten zur Auswahl geben, müsst ihre eine Wahl fällen, oder den dort beschreibenden Bedingungen folge Leisten.\\
Beginnen wir daher mit einer einfachen Frage!
\vfill


\textbf{Wie sehr sind ihnen komplexe Brettspiele vertraut?} \\
\def\dashfill{\cleaders\hbox{-}\hfill}
\hbox to \hsize{\dashfill\hfil}
\textbf{Seite: X} - Ja, ich finde durchaus das ich mich verschiedensten Brettspielen auskenne. Ich schätze ich werde die Regeln recht schnell verstehen.\\
\textbf{Seite: Y} - asdf

	%\end{flushleft}


\newpage


Text


\end{document}


















